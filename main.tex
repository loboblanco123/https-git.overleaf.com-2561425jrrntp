\documentclass[12pt,A4]{article}
\usepackage[utf8]{inputenc}
\usepackage[spanish, es-tabla]{babel}
\usepackage[version=3]{mhchem}
\usepackage[journal=jacs]{chemstyle}
\usepackage{amsmath}
\usepackage{amsfonts}
\usepackage{amssymb}
\usepackage{makeidx}
\usepackage{xcolor}
\usepackage[stable]{footmisc}
\usepackage[section]{placeins}
%Paquetes necesarios para tablas
\usepackage{longtable}
\usepackage{array}
\usepackage{xtab}
\usepackage{multirow}
\usepackage{colortab}
%Paquete para el manejo de las unidades
\usepackage{siunitx}
\sisetup{mode=text, output-decimal-marker = {,}, per-mode = symbol, qualifier-mode = phrase, qualifier-phrase = { de }, list-units = brackets, range-units = brackets, range-phrase = --}
\DeclareSIUnit[number-unit-product = \;] \atmosphere{atm}
\DeclareSIUnit[number-unit-product = \;] \pound{lb}
\DeclareSIUnit[number-unit-product = \;] \inch{"}
\DeclareSIUnit[number-unit-product = \;] \foot{ft}
\DeclareSIUnit[number-unit-product = \;] \yard{yd}
\DeclareSIUnit[number-unit-product = \;] \mile{mi}
\DeclareSIUnit[number-unit-product = \;] \pint{pt}
\DeclareSIUnit[number-unit-product = \;] \quart{qt}
\DeclareSIUnit[number-unit-product = \;] \flounce{fl-oz}
\DeclareSIUnit[number-unit-product = \;] \ounce{oz}
\DeclareSIUnit[number-unit-product = \;] \degreeFahrenheit{\SIUnitSymbolDegree F}
\DeclareSIUnit[number-unit-product = \;] \degreeRankine{\SIUnitSymbolDegree R}
\DeclareSIUnit[number-unit-product = \;] \usgallon{galón}
\DeclareSIUnit[number-unit-product = \;] \uma{uma}
\DeclareSIUnit[number-unit-product = \;] \ppm{ppm}
\DeclareSIUnit[number-unit-product = \;] \eqg{eq-g}
\DeclareSIUnit[number-unit-product = \;] \normal{\eqg\per\liter\of{solución}}
\DeclareSIUnit[number-unit-product = \;] \molal{\mole\per\kilo\gram\of{solvente}}
\usepackage{cancel}
%Paquetes necesarios para imágenes, pies de página, etc.
\usepackage{graphicx}
\usepackage{lmodern}
\usepackage{fancyhdr}
\usepackage[left=2cm,right=2cm,top=2cm,bottom=2cm]{geometry}

%Instrucción para evitar la indentación
%\setlength\parindent{0pt}
%Paquete para incluir la bibliografía
\usepackage[backend=bibtex,style=chem-acs,biblabel=dot]{biblatex}
\addbibresource{references.bib}

%Formato del título de las secciones

\usepackage{titlesec}
\usepackage{enumitem}
\titleformat*{\section}{\bfseries\large}
\titleformat*{\subsection}{\bfseries\normalsize}

%Creación del ambiente anexos
\usepackage{float}
\floatstyle{plaintop}
\newfloat{anexo}{thp}{anx}
\floatname{anexo}{Anexo}
\restylefloat{anexo}
\restylefloat{figure}

%Modificación del formato de los captions
\usepackage[margin=10pt,labelfont=bf]{caption}

%Paquete para incluir comentarios
\usepackage{todonotes}

%Paquete para incluir hipervínculos
\usepackage[colorlinks=true, 
            linkcolor = blue,
            urlcolor  = blue,
            citecolor = black,
            anchorcolor = blue]{hyperref}

%%%%%%%%%%%%%%%%%%%%%%
%Inicio del documento%
%%%%%%%%%%%%%%%%%%%%%%

\begin{document}
\renewcommand{\labelitemi}{$\checkmark$}

\renewcommand{\CancelColor}{\color{red}}

\newcolumntype{L}[1]{>{\raggedright\let\newline\\\arraybackslash}m{#1}}

\newcolumntype{C}[1]{>{\centering\let\newline\\\arraybackslash}m{#1}}

\newcolumntype{R}[1]{>{\raggedleft\let\newline\\\arraybackslash}m{#1}}

\begin{center}
	\textbf{\LARGE{WIKILYRICS - Repositorio de Letras de Canciones}}\\
	\vspace{7mm}
    \textbf{\large{INTEGRANTES:}}\\
		\textbf{\large{Quispe Barraza Jhilver Eloy}}\\
	\textbf{\large{Ripas Mamani Roger Dante}}\\
    \textbf{\large{Condori Luque Rodrigo}}\\
	\textbf{\large{Pucho Chuquicaña Jose}}\\
	\vspace{4mm}
	\textbf{\large{Sistemas Distribuidos}}\\
	\today
\end{center}

\vspace{7mm}

\section*{\centering Idea del Proyecto}
Se desarrollar un software de escritorio en principio que cuando se reproduce una canción esta pueda mostrar la letra de dicha canción en reproducción, posteriormente este aplicativo se desarrollara para una versión en Android.\\

Se decide tomar la decisión de hacer un repositorio de letras de canciones puesto que en el mercado existen software que si bien te dan la letra de la canción, no te dan la opción de agregar tu propia letra en caso de que esta no exista en los repositorios del software tal es el caso de MINILYRICS, el cual si bien nos da la opción de buscar nosotros la letra en su interfaz muchas veces no encontramos lo que deseamos, es ahi donde nos preguntamos si tengo yo la letra porque no puede haber una opción de subir letra y compartir con todosdas las personas.\\

En tanto en el ambito de las aplicaciones Android existe el problema de que no hay por interfaz la opcion de buscar letra o no podemos subir nuestra propia letra, tal es el caso de MUSIXMATCH.\\

Debido a esto es que escogimos hacer un WIKILYRICS, el cual sera un repositorio de letras de canciones.

\section*{\centering Justificación}
Se desarrollará un FTP ( File Transport Protocol) donde subiremos y descargaremos las letras de las canciones (lyrics), puesto que trabajaremos con una arquitectura cliente - servidor para la transferencia de letras, el cual será la base del proyecto a desarrollar.

\end{document}
